\chapter{Measure Force}

When slapping a tensioned Slackline with your finger you will hear a sound from the line multiple times. The idea is to calculate the pretension of the line from the frequency of that sound. The slapping of line results in  a mechanical wave propagating along the line and reflected at its ends. The propagation speed is depending on the tension of the line.

\section{Physical Background}

The propagation speed of a wave on a tensioned rope or slackline can be described with the following formula:

\begin{equation}
v = \sqrt{\frac{F}{\mu_0}}
\label{eqn:speed1}
\end{equation}

$v$ is the propagation speed, $F$ the force and $\mu_0$ the weight per unit length of the line. In between two sounds the mechanical wave travels from one anchor to the other and back, so it covers a distance two times the length $l$ of the line. If you measure the time of oscillation $t_{osc}$ the propagation speed can be calculated:

\begin{equation}
	v = \frac{2\cdot l}{t_{osc}}
	\label{eqn:speed2}
\end{equation}

If you combine equation \ref{eqn:speed1} and \ref{eqn:speed2} and solve for the force you will get the following equation:

\begin{equation}
	F_0 = \frac{4\cdot l^2\cdot \mu_0}{t_{osc}^2}
	\label{eqn:measureForce}
\end{equation}

$F_0$ is the pretension of the line. As you can see the weight and the length of the line have to be known. In the app the time of oscillation can be measured automatically with the microphone and some simple audio processing or manually by typing the heard sound pattern on a button.

\section{The Influence of the Stretch Characteristic}

During the tensioning of the line, the line also stretches. As the total weight of the line does not change this leads to a decrease of weight per unit length. Therefore equation \ref{eqn:measureForce} contains a systematic error on stretchy webbings. To take this into account a linear stretch behavior corresponding to the following formula is assumed.

\begin{equation}
	\frac{\Delta l}{l} = \alpha \cdot F
\end{equation}

$\alpha$ is the stretch coefficient and $F$ is the force on the line. The corrected weight per unit length is calculated to

\begin{equation}
	\mu = \mu_0 \cdot \frac{1}{1+\alpha\cdot F}
\end{equation}

If you replace $\mu_0$ in formula \ref{eqn:measureForce} with the corrected value $\mu$ and solve the equation for F you get the following formula:

\begin{equation}
	F = \sqrt{\frac{1}{4 \alpha^2} + \frac{\mu_0\cdot 4l^2}{\alpha \cdot t_{osc}^2}} - \frac{1}{2\alpha}
	\label{eqn:measureForceWithStretch}
\end{equation}

In the App this more exact formula is used whenever a stretch coefficient greater than zero is entered.
It can be shown that formula \ref{eqn:measureForceWithStretch} equals formula \ref{eqn:measureForce} for $\alpha\rightarrow 0$.